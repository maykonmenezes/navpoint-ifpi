%% Resumo
\begin{resumo}
A demanda por serviços de localização indoor (LBS) vem aumentando ao longo dos anos juntamente com
a expansão do mercado de smartphones. Há um crescente interesse no desenvolvimento de sistemas de navegação indoor para dispositivos móveis. Usuários de smartphones não conseguem navegar em áreas internas utilizando o GPS do aparelho pois ele não consegue fixar a localização exata de acordo com a função do GPS receiver. Essa é a grande razão da gigantesca demanda por informações de localização em tempo real de usuários desses portáteis. A função de GPS receiver tem uma poderosa precisão para localização em ambientes externos, mas em contrapartida o mesmo não se mostra eficaz por causa de atenuação de sinal.

Abordagens com informações pré-existentes sobre o ambiente (campos magnéticos, Bluetooth, WiFi) juntamente com dados de sensores dos smartphones (magnetômetro, giroscópio e acelerômetro) tem atraído mais e mais atenção por serem(.. a completar) . Este trabalho propõe um protótipo de aplicativo android de navegação indoor eficiente que une informações pré-existentes do ambiente com dados de sensores dos smartphones.


\vspace{\onelineskip}
\noindent
\textbf{Palavras-chaves}: Palavras chave.
\end{resumo}
