% ----------------------------------------------------------
% Conclusão
% ----------------------------------------------------------
\chapter{Conclusão}

Este trabalho apresentou a ferramenta SATA, uma ferramenta para manutenção de informações das avaliações de pacientes portadores do TEA, utilizando de seus resultados anteriores para, entre outras coisas, propor aos profissionais responsáveis informações sobre o sucesso ou insucesso do tratamento. Para isso, foram consultados diversos profissionais médicos e pais de pacientes portadores de TEA que elencavam as dificuldades existentes no atual método de tratamento sem esta ferramenta e no que ela poderia propor aos usuários.

Para os testes do sistema, foram criados personagens fictícios, com base em portadores anônimos de TEA. Com esses personagens fictícios, foram realizadas simulações de funcionamento da aplicação. Os resultados mostraram a capacidade de avaliação e acompanhamento de acordo com os padrões estabelecidos pela técnica de rastreio utilizada.

Por fim, o trabalho também contribui a integração entre as pessoas envolvidas com os pacientes (profissionais e responsáveis), permitindo uma troca de informação de maneira facilitada entre os profissionais médicos e uma transparência na exibição dos resultados aos responsáveis.

\section{Trabalhos Futuros}

Existem diversas outras técnicas de rastreio e avaliação do tratamento de autismo. Diante disso, poderão ser desenvolvidas novas versões dessa ferramenta com suporte a essas novas técnicas assim como já se tem com a CARS. Dentre as técnicas de rastreio que podem ser utilizadas em versões futuras do SATA, estão:

\begin{lista}
	\item Vineland~\cite{ferreira:2014};
	\item M-CHAT (\emph{Modified Checklist for Autism in Toddlers})~\cite{robins:2001};
	\item AGF (Avaliação Global de Funcionamento)~\cite{alcantara:2004}.
\end{lista}

\section{Publicações}

O SATA foi aceito e será divulgado por meio de artigo resumido no XXVIII Simpósio Brasileiro de Informática na Educação, que acontecerá na cidade de Recife-PE, no período de 30 de outubro a 02 de novembro de 2017.