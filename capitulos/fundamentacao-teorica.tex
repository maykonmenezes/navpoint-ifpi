% ----------------------------------------------------------
% Fundamentação Teórica
% ----------------------------------------------------------
\chapter{Fundamentação Teórica} \label{fundamentacao}
Para melhor compreender a finalidade desse trabalho, é necessário entender alguns conceitos relacionados a sistemas de navegação e localização indoor.

\section{Dispositivos Móveis}
\subsection{Smartphones}
\subsection{Sensores}
\begin{lista}
   \item Magnetometro;
   \item Barometro;
\end{lista}
\section{Tecnicas de Localização Indoor}
Todo o sistema LBS deve focar-se na localização geográfica dos dispositivos móveis.
Segundo \cite{surveywireless}, diferentes tipos de aplicações podem exigir diferentes tipos de informação sobre a localização. Com base no que os autores de  \cite{locationsystems} afirmam, existem quatro
tipos diferentes, a localização física, a simbólica, a absoluta e a relativa.

Segundo \cite{locationsystems}, a localização física é expressa na forma de coordenadas que identificam
um ponto num mapa a duas dimensões ou a três dimensões. Este tipo de localização utiliza, em quase todos os sistemas implementados, um sistema de coordenadas baseado em
graus/minutos/segundos ou grau minutos decimais ou ainda o sistema UTM (Universal
Transverse Mercator). No seguimento do mesmo artigo, os autores definem que a localização simbólica expressa a componente da localização numa linguagem natural, como
por exemplo o escritório, ou o terceiro piso, ou o quarto, entre outros que possam definir
a localização. A localização absoluta é utilizada a partir de uma grelha de referência onde
todos os objetos estão localizados e a localização relativa é sempre baseada em pontos de
referência que transmitem um local aproximado do objeto a ser localizado. De todos os
tipos de localização que existem, torna-se claro que o tipo de localização mais utilizado e
em que, praticamente, todos os sistemas se baseiam é o tipo de localização física.

\subsection{Triangulação}
\subsection{Location Figerprinting}
\subsection{Proximidade}



