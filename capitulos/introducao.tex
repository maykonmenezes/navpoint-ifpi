% ----------------------------------------------------------
% Introdução
% ----------------------------------------------------------
\chapter{Introdução}

Com o rápido desenvolvimento da comunicação através de portáteis e a difusão de tecnologias de computação em todas as áreas, a necessidade de se obter serviços de localização e navegação está rapidamente crescendo. Melhorias dramáticas em performance dos padrões de comunicação moveis tem impulsionado a tecnologia móvel a se tornar o meio mais rápido de comunicação de todos os tempos. Os custos de infraestrutura com redes moveis também caíram drasticamente, enquanto performance só tem melhorado.

Na ultima década, temos visto grandes melhorias na redução de tamanho de hardwares, a chegada de muitas novas tecnologias, como redes wireless, baterias com grandes capacidades, chips de alta performance etc, que fazem do smartphone uma ferramenta poderosa. Essas tecnologias permitiram que os fabricantes construíssem dispositivos moveis que podem ser carregados por ai com a mesma performance de computadores tradicionais. Os benefícios de toda essa tecnologia embarcada pode ser aproveitada pelos chamados serviços baseados em localização. Aplicações que guiam usuários, que se comportam diferente baseado na localização ou contexto do usuário, ou melhor, que conseguem navegar o usuário através de um local e oferecer informações baseado em onde ele está são atualmente “trend topic”  em pesquisa e são considerados como um mercado promissor.

Atualmente, o Sistema de Posicionamento Global (GPS) oferece informação de localização precisa e confiável para serviços de localização. O GPS não pode ser usado efetivamente dentro de um ambiente interno porque ocorre uma degradação de sinal. Por outro lado, informações pré-existentes do ambiente e sensores que possibilitam localizar como acelerômetros, giroscópios, magnetômetro, WiFi, câmeras etc podem ser usados por serviços baseados em localização para navegação e posicionamento em ambientes internos.

Nesse trabalho, discutiremos sobre um sistema de navegação e localização feito utilizando dados pré-existentes do ambiente e sensores do smartphone. Mas antes de prosseguir precisamos saber, quais são as tecnologias disponíveis para navegação e localização internas disponíveis? E porque utilizar dados pré-existentes do ambientes e sensores do smartphone?



\section{Justificativa}
Justificar a escolha do tema.



\section{Objetivos}
Nesta seção os objetivos gerais e especificos são enunciados.
\subsection{Objetivo Geral}
O objetivo geral deste trabalho é apresentar uma solução para localização e navegação de um ponto a outro em ambiente fechado. O desenvolvimento do projeto almeja prover um protótipo de aplicativo de navegação interna, mas que utiliza apenas informações pré-existentes do ambiente e os sensores do smartphone. A navegação acontece em rotas já pré-estabelecidas do ambiente mapeado, servindo como um guia dos usuários por destinos comuns e trajetos mais rápidos, visando prover uma maneira de navegar rapidamente entre locais do ambiente interno.



\subsection{Objetivos Específicos}
\begin{lista}
  \item Estabelecer rotas mais rápidas para o navegação dado a localização atual do usuário e o seu destino final;
  \item Navegar o usuário através do instituto de um ponto ao outro;
  \item Notificar o usuário com informações relevantes do ambiente baseado na sua localização atual;
\end{lista}

\textbf{RESUMO}
O presente capítulo apresenta uma contextualização sobre o problema tratado neste
trabalho e a justificativa de tal assunto, que pode ser resumida como sendo a proposta de um sistema de navegação e localização projetado para guiar a locomoção de pessoas dentro do Instituto.  Ao final do
capítulo, foram detalhados os objetivos gerais e específicos do trabalho.
Os próximos capítulos estão organizados da seguinte forma:


