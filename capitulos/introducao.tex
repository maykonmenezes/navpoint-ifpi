% ----------------------------------------------------------
% Introdução
% ----------------------------------------------------------
\chapter{Introdução}

Com o rápido desenvolvimento da comunicação através de portáteis e a difusão de tecnologias de computação em todas as áreas, a necessidade de se obter serviços de localização e navegação está rapidamente crescendo. Melhorias dramáticas em performance dos padrões de comunicação moveis tem impulsionado a tecnologia móvel a se tornar o meio mais rápido de comunicação de todos os tempos. Os custos de infraestrutura com redes moveis também caíram drasticamente, enquanto performance só tem melhorado.

Na ultima década, temos visto grandes melhorias na redução de tamanho de hardwares, a chegada de muitas novas tecnologias, como redes wireless, baterias com grandes capacidades, chips de alta performance etc, que fazem do smartphone uma ferramenta poderosa. Essas tecnologias permitiram que os fabricantes construíssem dispositivos moveis que podem ser carregados por ai com a mesma performance de computadores tradicionais. Os benefícios de toda essa tecnologia embarcada pode ser aproveitada pelos chamados serviços baseados em localização. Aplicações que guiam usuários, que se comportam diferente baseado na localização ou contexto do usuário, ou melhor, que conseguem navegar o usuário através de um local e oferecer informações baseado em onde ele está são atualmente “trend topic”  em pesquisa e são considerados como um mercado promissor.

Atualmente, o Sistema de Posicionamento Global (GPS) oferece informação de localização precisa e confiável para serviços de localização. O GPS não pode ser usado efetivamente dentro de um ambiente interno porque ocorre uma degradação de sinal. Por outro lado, informações pré-existentes do ambiente e sensores que possibilitam localizar como acelerômetros, giroscópios, magnetômetro, WiFi, câmeras etc podem ser usados por serviços baseados em localização para navegação e posicionamento em ambientes internos.

Nesse trabalho, discutiremos sobre um sistema de navegação e localização feito utilizando dados pré-existentes do ambiente e sensores do smartphone. Mas antes de prosseguir precisamos saber, quais são as tecnologias disponíveis para navegação e localização internas disponíveis? E porque utilizar dados pré-existentes do ambientes e sensores do smartphone?



\section{Justificativa}
A situação atual dos serviços de navegação indoor, inclui dificuldades de varias ordens, mas as mais comuns são as relacionadas a custo com equipamentos e implementação. Com a impossibilidade de utilização do GPS em ambientes fechados, que é um equipamento que atualmente é de fácil obtenção, o custo para promover meios de localizar e navegar nesse cenário geralmente se tornam caros. Existem atualmente vários sensores para promover localização indoor, dentre eles estão o sensores ultrassónicos, sensores de infravermelho, magnetômetros e sensores de radio. Mas todos eles requerem uma infraestrutura eletrônica permanente para facilitar as medições, e objetos localizáveis que dependem dessa infraestrutura precisam de sensores e atuadores especiais. Existem vários problemas práticos com esse tipo de abordagem, como consumo de energia, fiação e custos com infraestrutura em geral inibem a implantação dessa tecnologia em prédios e outros ambientes fechados.

Sistemas de navegação baseados em GPS tem se tornado bastante popular, principalmente porque permite que as pessoas explorem áreas desconhecidas rapidamente. Entretanto, o GPS funciona somente para ambientes externos, os links de satelite necessarios são bloqueados por conta da estrutura e se tornam não confiaveis dentro dos predios.

Neste contexto, o sistema para navegação e localização indoor apresentado neste trabalho ganha vantagem por associar informações pré-existentes no ambiente com dados dos sensores dos smartphones do usuarios, provendo uma solução viavel e de baixo custo.



\section{Objetivos}
Nesta seção os objetivos gerais e especificos são enunciados.
\subsection{Objetivo Geral}
O objetivo geral deste trabalho é apresentar uma solução de baixo custo para localização e navegação de pessoas de forma rápida e eficiente por ambientes internos. O desenvolvimento do projeto almeja prover um protótipo de aplicativo de navegação interna, mas que utiliza apenas informações pré-existentes do ambiente e os sensores do smartphone. A navegação acontece em rotas já pré-estabelecidas do ambiente mapeado, servindo como um guia dos usuários por destinos comuns e trajetos mais rápidos, visando prover uma maneira de navegar rapidamente entre locais do ambiente interno.



\subsection{Objetivos Específicos}
\begin{lista}
  \item Estabelecer rotas mais rápidas para o navegação dado a localização atual do usuário e o seu destino final;
  \item Navegar o usuário através do instituto de um ponto ao outro;
  \item Notificar o usuário com informações relevantes do ambiente baseado na sua localização atual;
\end{lista}

\section*{RESUMO}
O presente capítulo apresenta uma contextualização sobre o problema tratado neste
trabalho e a justificativa de tal assunto, que pode ser resumida como sendo a proposta de um sistema de navegação e localização projetado para guiar a locomoção de pessoas dentro do Instituto.  Ao final do
capítulo, foram detalhados os objetivos gerais e específicos do trabalho.

Os próximos capítulos estão organizados da seguinte forma:

\begin{lista}
  \item \textbf{Fundamentação Teórica:} Neste capítulo são apresentados todos os conceitos teóricos utilizados no desenvolvimento da solução proposta no presente trabalho;
 \item \textbf{Trabalhos Relacionados:} Neste capítulo são apresentados os principais trabalhos que se relacionam a proposta de solução para navegação em ambientes fechados ;
  \item \textbf{NavPoint:} Capítulo dedicado a apresentação da solução implementada, detalhando funcionalidades, utilização e demais detalhes da implementação;
 \item \textbf{Protótipo e Testes:} Capítulo dedicado a apresentação de alguns testes feitos com a ferramenta, bem como o cenário utilizado para os mesmos;
  \item \textbf{Conclusão:} Neste capítulo é feita a conclusão do trabalho dado seus objetivos propostos e são listados os trabalhos futuros para melhorar e/ou expandir a utilização da solução.
\end{lista}


