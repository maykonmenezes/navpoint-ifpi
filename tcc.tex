%%%%%%%%%%%%%%%%%%%%%%%%%%%%%%%%%%%%%%%%%%%%%%%%%%%%%%%
%%      TCC de Lucas Rodrigues de Carvalho, 2016     %%
%%%%%%%%%%%%%%%%%%%%%%%%%%%%%%%%%%%%%%%%%%%%%%%%%%%%%%%

% Alterado por Prof. Fernando C.B.G. Santana
% A simplificação visa melhorar a navegação e edição

%% Preâmbulo (configurações, pacotes e tudo mais)

\documentclass[
	%article,			% Define que este será um artigo (e não uma tese/monografia/relatório)
	12pt,				% Fonte: 12pt
	oneside,			% Impressão: oneside = 1 face, twoside = 2 faces (frente-e-verso)
    %openright,			% capítulos começam em página ímpar (use apenas se usar "twoside")
	a4paper,			% Tamanho do Papel: A4
    chapter=TITLE,		% Todos os capítulos devem ficam em caixa alta
    section=TITLE,		% Todas as seções devem ficar em caixa alta
	english,			% Adiciona Idioma para Hifenização: Inglês
    %spanish,			% Adiciona Idioma para Hifenização: Espanhol
    %french,			% Adiciona Idioma para Hifenização: Francês
	brazil				% Adiciona Idioma para Hifenização: Português do Brasil (o último idioma se torna o principal do documento)
]{abntex2}				% Utilizar ABNTeX2

%% Tipografia
%% Abra este arquivo e selecione uma das opções de fonte nele. A padrão é Times.
\input{configuracoes/tipografia}

%% Pacotes usados pelo documento (se não entender não mexa, hehehe)
\usepackage{courier}                    % Permite a utilização da fonte Courier (para códigos-fonte)
\usepackage[T1]{fontenc}				% Seleção de códigos de fonte.
\usepackage[utf8]{inputenc}				% Codificação do documento (conversão automática dos acentos)
\usepackage{indentfirst}				% Indenta o primeiro parágrafo de cada seção.
\usepackage{nomencl} 					% Usado pela Lista de símbolos
\usepackage{color}						% Controle das cores
\usepackage{graphicx}					% Inclusão de gráficos
\let\newfloat\undefined
\usepackage{float}						% Melhorias para posicionamento de gráficos e tabelas
\usepackage{microtype} 					% Melhorias na justificação
\usepackage{lastpage}   		        % Dá acesso ao número da última página do documento
\usepackage{booktabs}					% Comandos para tabelas
\usepackage{multirow, array}			% Múltiplas linhas e colunas em tabelas
\usepackage[hyphenbreaks]{breakurl}		% Hifenação para URLs no texto
\usepackage[table,xcdraw]{xcolor}       % Cores para algumas tabelas especiais
\usepackage[brazilian,hyperpageref]{backref}	 % Inclui nas Referências as páginas onde há as citações
\usepackage{simplecd}                   % Pacote para gerar capa do CD
\usepackage[final]{pdfpages}            % Pacote para incluir um PDF dentro de outro (ficha catalográfica)

%% Adiciona as alterações do ABNTeX-IFPI
\usepackage{abntex-ifpi/abntex-ifpi}	% Modificações do ABNTeX para o IFPI
\usepackage{abntex-ifpi/tikz-uml}	    % Pacote Tikz UML para criar UML no LaTeX

%% Metadados
\input{configuracoes/pdf}
%% Metadados
\input{configuracoes/metadados}
%% Configuração do "Citado nas páginas"
\input{configuracoes/citacoes}
%% Cores
\input{configuracoes/cores}
%% Espaçamentos
\input{configuracoes/espacamentos}

%% Início do Documento
\begin{document}

%% Documento será escrito em Português do Brasil
\selectlanguage{brazil}

%% Elementos Pré Textuais
%% ----------------------
%% 
%% Segundo o manual do IFPI, eles devem ser os seguintes (nessa ordem):
%%  1. Capa (obrigatório)
%%  2. Folha de rosto (obrigatório)
%%  3. Errata (opcional)
%%  4. Folha de aprovação (obrigatório)
%%  5. Dedicatória (opcional)
%%  6. Agradecimentos (opcional)
%%  7. Epígrafe (opcional)
%%  8. Resumo (obrigatório)
%%  9. Abstract/Resumo em outra língua (obrigatório)
%% 10. Lista de Ilustrações (opcional)
%% 11. Lista de Tabelas (opcional)
%% 12. Lista de Abreviaturas e Siglas (opcional)
%% 13. Lista de Símbolos (opcional)
%% 14. Sumário (obrigatório)
%% %%%%%%%%%%%%%%%%%%%%%%%%%%%%%%%%%%%%%%%%% %%

%% 01: Capa
\imprimircapa

%% 02: Folha de Rosto
%% OBS: O asterisco indica que haverá ficha bibliográfica (só funciona para impressão frente-e-verso)
\imprimirfolhaderosto*

%% Ficha Catalográfica (acho que é melhor adicionar via \includepdf depois)
%\input{pre-textual/ficha-catalografica}

%% 03: Errata
%\input{pre-textual/errata}

%% 04: Folha de Aprovação
%\imprimirfolhadeaprovacao
%% Use esta se forem 4 membros na banca:
\imprimirfolhadeaprovacaoduascolunas

%% 05: Dedicatória
%%% Dedicatória do seu trabalho
\begin{dedicatoria}
	%% Empura o texto a seguir para a parte de baixo da página
	\vspace*{\fill}
    
    %% Alinhado a Direita
    \center
    \begin{flushright}
    	Dedicatória.
    \end{flushright}
    
    %% Descomente a linha seguir para deixar o texto centralizado verticalmente na página
    %% Lembre de comentar o "\begin{}" e "\end{}" acima para centralizar o texto da dedicatória.
	%\vspace*{\fill}
\end{dedicatoria}


%% 06: Agradecimentos
\input{pre-textual/agradecimentos}

%% 07: Epígrafe
%% Epígrafe
%% Uma frase que lhe inspira ou a qual lhe inspirou a fazer este trabalho
\begin{epigrafe}
\vspace*{\fill}
\begin{flushright}
\emph{``Epigrafe'' \\ (Autor)}
\end{flushright}
\end{epigrafe}


%% 08: Resumo
%% Resumo
\begin{resumo}
A demanda por serviços de localização indoor (LBS) vem aumentando ao longo dos anos juntamente com
a expansão do mercado de smartphones. Há um crescente interesse no desenvolvimento de sistemas de navegação indoor para dispositivos móveis. Usuários de smartphones não conseguem navegar em áreas internas utilizando o GPS do aparelho pois ele não consegue fixar a localização exata de acordo com a função do GPS receiver. Essa é a grande razão da gigantesca demanda por informações de localização em tempo real de usuários desses portáteis. A função de GPS receiver tem uma poderosa precisão para localização em ambientes externos, mas em contrapartida o mesmo não se mostra eficaz por causa de atenuação de sinal.

Abordagens com informações pré-existentes sobre o ambiente (campos magnéticos, Bluetooth, WiFi) juntamente com dados de sensores dos smartphones (magnetômetro, giroscópio e acelerômetro) tem atraído mais e mais atenção por serem(.. a completar) . Este trabalho propõe um protótipo de aplicativo android de navegação indoor eficiente que une informações pré-existentes do ambiente com dados de sensores dos smartphones.


\vspace{\onelineskip}
\noindent
\textbf{Palavras-chaves}: Palavras chave.
\end{resumo}


%% 09: Abstract/Resumo em língua estrangeira
%% Abstract (configurado para língua inglesa)
\begin{resumo}[Abstract]			% Título do Resumo (Abstract = Resumo em inglês)
\begin{otherlanguage*}{english}		% Língua do texto
The demand for Indoor Location Based Services
(LBS) is increasing over the past years as
smartphone market expands. There's a growing
interest in developing efficient and reliable
indoor positioning systems for mobile devices.
Smartphone users can get their fixed locations
according to the function of the GPS receiver.
This is the primary reason why there is a huge
demand for real-time location information of
mobile users. However, the GPS receiver is
often not effective in indoor environments due
to signal attenuation, even as the major
positioning devices have a powerful accuracy
for outdoor positioning.

Using preexisting information about the environment(magnetics, bluetooth, WiFi) among with smartphone's sensors data (compass, gyroscope, accelerator) attract more and more attention due to. This paper proposes an efficient and reliable android application prototype of indoor navagation.

\vspace{\onelineskip}
\noindent
\textbf{Key-words}: Palavras-chave em língua estrangeira.
\end{otherlanguage*}
\end{resumo}

%% Exemplo de resumo em francês
%\begin{resumo}[Résumé]
% \begin{otherlanguage*}{french}
%    Il s'agit d'un résumé en français.
% 
%   \textbf{Mots-clés}: latex. abntex. publication de textes.
% \end{otherlanguage*}
%\end{resumo}

%% Exemplo de resumo em Espanhol
%\begin{resumo}[Resumen]
% \begin{otherlanguage*}{spanish}
%   Este es el resumen en español.
%  
%   \textbf{Palabras clave}: latex. abntex. publicación de textos.
% \end{otherlanguage*}
%\end{resumo}
% ---


%% 10: Lista de Ilustrações
\input{pre-textual/lista-de-ilustracoes}

%% 11: Lista de Tabelas
\input{pre-textual/lista-de-tabelas}

%% 12: Lista de Abreviaturas e Siglas
%% Lista de Siglas
\begin{siglas}
  \item[TEA] Transtorno de Espectro Autista
  \item[CARS] \emph{Childhood Autism Rating Scale}
  \item[SATA] Sistema de Acompanhamento do Tratamento de Autismo
  \item[ABA] \emph{Applied Behavior Analysis}
  \item[M-CHAT] \emph{Modified Checklist for Autism in Toddlers}
  \item[AGF] Avaliação Global de Funcionamento
\end{siglas}

%% 13: Lista de Símbolos
% \input{pre-textual/simbolos}

%% 14: Sumário (o asterisco retira o próprio sumário do sumário)
\pdfbookmark[0]{\contentsname}{toc}
\tableofcontents*
\cleardoublepage

%% Indica que a partir daqui ficarão os elementos textuais (TCC em si)
\textual
%% Inclui os capítulos do TCC
% ----------------------------------------------------------
% Introdução
% ----------------------------------------------------------
\chapter{Introdução}

Com o rápido desenvolvimento da comunicação através de portáteis e a difusão de tecnologias de computação em todas as áreas, a necessidade de se obter serviços de localização e navegação está rapidamente crescendo. Melhorias dramáticas em performance dos padrões de comunicação moveis tem impulsionado a tecnologia móvel a se tornar o meio mais rápido de comunicação de todos os tempos. Os custos de infraestrutura com redes moveis também caíram drasticamente, enquanto performance só tem melhorado.

Na ultima década, temos visto grandes melhorias na redução de tamanho de hardwares, a chegada de muitas novas tecnologias, como redes wireless, baterias com grandes capacidades, chips de alta performance etc, que fazem do smartphone uma ferramenta poderosa. Essas tecnologias permitiram que os fabricantes construíssem dispositivos moveis que podem ser carregados por ai com a mesma performance de computadores tradicionais. Os benefícios de toda essa tecnologia embarcada pode ser aproveitada pelos chamados serviços baseados em localização. Aplicações que guiam usuários, que se comportam diferente baseado na localização ou contexto do usuário, ou melhor, que conseguem navegar o usuário através de um local e oferecer informações baseado em onde ele está são atualmente “trend topic”  em pesquisa e são considerados como um mercado promissor.

Atualmente, o Sistema de Posicionamento Global (GPS) oferece informação de localização precisa e confiável para serviços de localização. O GPS não pode ser usado efetivamente dentro de um ambiente interno porque ocorre uma degradação de sinal. Por outro lado, informações pré-existentes do ambiente e sensores que possibilitam localizar como acelerômetros, giroscópios, magnetômetro, WiFi, câmeras etc podem ser usados por serviços baseados em localização para navegação e posicionamento em ambientes internos.

Nesse trabalho, discutiremos sobre um sistema de navegação e localização feito utilizando dados pré-existentes do ambiente e sensores do smartphone. Mas antes de prosseguir precisamos saber, quais são as tecnologias disponíveis para navegação e localização internas disponíveis? E porque utilizar dados pré-existentes do ambientes e sensores do smartphone?



\section{Justificativa}
Justificar a escolha do tema.



\section{Objetivos}
Nesta seção os objetivos gerais e especificos são enunciados.
\subsection{Objetivo Geral}
O objetivo geral deste trabalho é apresentar uma solução para localização e navegação de um ponto a outro em ambiente fechado. O desenvolvimento do projeto almeja prover um protótipo de aplicativo de navegação interna, mas que utiliza apenas informações pré-existentes do ambiente e os sensores do smartphone. A navegação acontece em rotas já pré-estabelecidas do ambiente mapeado, servindo como um guia dos usuários por destinos comuns e trajetos mais rápidos, visando prover uma maneira de navegar rapidamente entre locais do ambiente interno.



\subsection{Objetivos Específicos}
\begin{lista}
  \item Estabelecer rotas mais rápidas para o navegação dado a localização atual do usuário e o seu destino final;
  \item Navegar o usuário através do instituto de um ponto ao outro;
  \item Notificar o usuário com informações relevantes do ambiente baseado na sua localização atual;
\end{lista}

\textbf{RESUMO}
O presente capítulo apresenta uma contextualização sobre o problema tratado neste
trabalho e a justificativa de tal assunto, que pode ser resumida como sendo a proposta de um sistema de navegação e localização projetado para guiar a locomoção de pessoas dentro do Instituto.  Ao final do
capítulo, foram detalhados os objetivos gerais e específicos do trabalho.
Os próximos capítulos estão organizados da seguinte forma:



% ----------------------------------------------------------
% Fundamentação Teórica
% ----------------------------------------------------------
\chapter{Fundamentação Teórica} \label{fundamentacao}
Para melhor compreender a finalidade desse trabalho, é necessário entender alguns conceitos relacionados ao tema escolhido.

% ----------------------------------------------------------
% Trabalhos Relacionados
% ----------------------------------------------------------
\chapter{Trabalhos Relacionados}

Este capítulo apresenta trabalhos cujo conteúdo se relacione ao tema escolhido.

% ----------------------------------------------------------
% Trabalhos Relacionados
% ----------------------------------------------------------
\chapter{Metodologia}

Este capítulo apresenta a metodologia utilizada para o desenvolvido do projeto.

% ----------------------------------------------------------
% Trabalhos Relacionados
% ----------------------------------------------------------
\chapter{Cronograma}

Este capítulo apresenta o cronograma para o desenvolvimento do projeto.

%\input{capitulos/trabalhos-futuros.tex}

%% Finalizações para o PDF
\bookmarksetup{startatroot}

%% Indica ao LaTeX que a partir deste ponto ficarão os elementos pós-textuais (Anexos... etc.)
\postextual

%% 01: Referências bibliográficas
\bibliography{pos-textual/bibliografia}

%% 02: Glossário
\input{pos-textual/glossario}

%% 03: Apêndices
\input{pos-textual/apendices}

%% 04: Anexos
%\input{pos-textual/anexos}

%% 05: Índices
\input{pos-textual/indices}

\end{document}
